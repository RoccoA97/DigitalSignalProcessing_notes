\providecommand{\main}{../../main}
\providecommand{\figpath}[1]{\main/../lessons/#1}
\documentclass[../../main/main.tex]{subfiles}



\begin{document}

\newdate{date}{15}{12}{2020}
\marginpar{\textbf{Lecture 21.} \\ \displaydate{date}.}

\section{Comb filters}
\marginpar{Comb filters}
The simple filters discussed so far are characterized either by a single passband and/or a single stopband. There are applications where these kind of stuff is not sufficient, so filters with multiple passbands and stopbands are required. The so called \textbf{comb filters} are an example of such instruments.

A comb filter is an LTI digital filter such that:
\begin{itemize}
    \item its output to a scaled sum of input digital signals is equal to the scaled sum of the outputs to every one of these input signals (i.e., the filter satisfies the \textbf{superposition principle});
    \item for any input signal that has a given delay, the output undergoes the same delay as the input.
\end{itemize}
\marginpar{Features in general form}
In its most general form, a comb filter has a frequency response that is a periodic function of \( \omega \), with a period \( \frac{2\pi}{L} \), where \( L \) is a positive integer.
Moreover, if \( H(z) \) is a filter with a single passband and/or a single stopband, a comb filter can be easily generated from it by replacing each delay in its realization with \( L \) delays, resulting in a structure with a transfer function given by:
\begin{equation}
    G(z)
    =
    H(z^{L})
    \label{eq:L21_S05_1}
\end{equation}
In particular, if \( \abs{H(e^{j\omega})} \) exhibits a peak at \( \omega_{p} \), then \( \abs{G(e^{j\omega})} \) will exhibits \( L \) peaks at \( \frac{\omega_{p}k}{L} \), with \( 0 \le k \le L-1 \), in the frequency range \( 0 \le \omega \le 2\pi \).
Likewise, \( \abs{H(e^{j\omega})} \) has a notch at \( \omega_{0} \), then \( \abs{G(e^{j\omega})} \) will have \( L \) notches at \( \frac{\omega_{0}k}{L} \), with \( 0 \le k \le L-1 \), in the frequency range \( 0 \le \omega \le 2\pi \).

Another important thing to remark is that a comb filter can be generated from either an FIR or an IIR prototype filter.

\marginpar{Examples of comb filters}
\begin{example}{Comb filters from FIR/IIR prototype filter}{}
    The comb filter generated from the prototype highpass FIR filter:
    \begin{equation}
        H_{0}(z) = \frac{1}{2} \qty(1 + z^{-1})
        \label{eq:L21_S07_1}
    \end{equation}
    has a transfer function:
    \begin{equation}
        G_{0}(z)
        =
        H_{0}(z^{L})
        =
        \frac{1}{2} \qty(1 + z^{-L})
        \label{eq:L21_S07_2}
    \end{equation}
    \( \abs{G_{0}(e^{j\omega})} \) has \( L \) notches at \( \omega = \frac{(2k+1)\pi}{L} \) and \( L \) peaks at \( \omega = \frac{2\pi k}{L} \), with \( 0 \le k \le L-1 \), in the frequency range \( 0 \le \omega < 2\pi \).

    \begin{center}
        \includegraphics[width=0.5\textwidth]{\figpath{21}/21_images/S07_1.pdf}
    \end{center}

    On the other hand, the comb filter generated from the prototype highpass FIR filter:
    \begin{equation}
        H_{1}(z)
        =
        \frac{1}{2} (1 - z^{-1})
        \label{eq:L21_S08_1}
    \end{equation}
    has a transfer function:
    \begin{equation}
        G_{0}(z)
        =
        H_{1}(z^{L})
        =
        \frac{1}{2} (1 - z^{-L})
        \label{eq:L21_S08_2}
    \end{equation}
    \( \abs{G_{0}(e^{j\omega})} \) has \( L \) peaks at \( \omega = \frac{(2k+1)\pi}{L} \) and \( L \) notches at \( \omega = \frac{2\pi k}{L} \), with \( 0 \le k \le L-1 \), in the frequency range \( 0 \le \omega < 2\pi \).

    \begin{center}
        \includegraphics[width=0.5\textwidth]{\figpath{21}/21_images/S08_1.pdf}
    \end{center}
\end{example}

Depending on the application, comb filters with other types of periodic magnitude responses can be easily generated by appropriately choosing the prototype average filter.

\begin{example}{\boldmath Comb filter from \( M \)-point moving average}{}
    The \( M \)-point moving average filter:
    \begin{equation}
        H(z)
        =
        \frac{1 - z^{-M}}{M(1 - z^{-1})}
        \label{eq:L21_S09_1}
    \end{equation}
    can be used as a prototype. This filter has a peak magnitude at \( \omega = 0 \), and \( M-1 \) notches at \( \omega = \frac{2\pi\ell}{M} \), with \( 1 \le \ell \le M-1 \). The corresponding comb filter will have a transfer function:
    \begin{equation}
        G(z)
        =
        \frac{1 - z^{-LM}}{M(1 - z^{-L})}
        \label{eq:L21_S10_1}
    \end{equation}
    whose magnitude has \( L \) peaks at \( \omega = \frac{2\pi k}{LM} \), with \( 1 \le k \le L(M-1) \). By choosing \( L \) and \( M \) appropriately, peaks and notches can be created at desired locations.
\end{example}





\section{Digital filter structures}
\marginpar{Digital filter structures}
Now, before going on with the discussion, let us return to basic concepts of an LTI discrete-time system. Let us consider:
\begin{equation}
    y[n]
    =
    \sum_{k=-\infty}^{\infty} x[k]h[n-k]
    =
    \sum_{k=-\infty}^{\infty} x[n-k]h[n]
    \label{eq:L21_S11_1}
\end{equation}
The formula in Eq. \ref{eq:L21_S11_1} expresses the \( n^{\text{th}} \) output sample as a convolution sum of the input with the impulse response of the system. This can be considered as the characterization of the system and, therefore, the convolution sum can in principle be used to implement the system. It involves additions, multiplications and delays, which are simple operations. In particular, the input-output relation involves a finite sum of products:
\begin{equation}
    y[n]
    =
    - \sum_{k=1}^{N} d_{k}y[n-k] + \sum_{k=0}^{M} p_{k}x[n-k]
    \label{eq:L21_S12_1}
\end{equation}
On the other hand, an FIR system can be implemented using the convolution sum, which is a finite sum of products:
\begin{equation}
    y[n]
    =
    \sum_{k=0}^{N} h[k]x[n-k]
    \label{eq:L21_S12_2}
\end{equation}
Now, the actual implementation of an LTI digital filter can be either in software or hardware form, depending on the application. So, in either case, the signal variables and the filter coefficients cannot be represented with infinite precision.

A direct implementation based on either the difference equation or the finite convolution sum may not provide satisfactory performance due to the finite precision arithmentic. Thus, it is of practical interest to develop alternate realizations and choose the structure that provides satisfactory performances under this framework.

In this context, a \textbf{structural representation} using interconnected basic buiding blocks is the first step\marginpar{Structural representation} in the hardware or software implementation of an LTI digital filter.
The structural representation provides the key relations between some pertinent internal variables with the input and output that in turn provides the key to the implementation.



\subsection{Block diagram representation}
\marginpar{Block diagram representation of computational algorithms}
We have seen that, in time domain, the input-output relations of an LTI digital filter is given by the convolution sum in Eq. \ref{eq:L21_S11_1} or by the linear constant coefficient difference Eq. \ref{eq:L21_S12_1}.
For the implementation, the input-output relationship must be descirbed by a \textbf{valid computational algorithm}.
To illustrate what we are meaning, let us consider the causal first-order LTI digital filter showed in Figure \ref{fig:L21_S17_1}.

\begin{figure}[!h]
    \centering
    \includegraphics[width=0.5\textwidth]{\figpath{21}/21_images/S17_1.pdf}
    \caption{\label{fig:L21_S17_1} Block diagram representation of a causal first-order LTI digital filter.}
\end{figure}

\marginpar{Example with a causal first-order LTI digital filter}%
This filter is described by the difference equation:
\begin{equation}
    y[n]
    =
    -d_{1}y[n-1] + p_{0}x[n] + p_{1}x[n-1]
    \label{eq:L21_S18_1}
\end{equation}
Using Eq. \ref{eq:L21_S18_1}, we can compute \( y[n] \) for \( n \ge 0 \) knowing the initial condition \( y[-1] \) and the input \( x[n] \) for \( n \ge -1 \):
\begin{align}
    y[0] &= -d_{1}y[-1] + p_{0}x[0] + p_{1}x[-1]    \\
    y[1] &= -d_{1}y[ 0] + p_{0}x[1] + p_{1}x[ 0]    \\
    y[2] &= -d_{1}y[ 1] + p_{0}x[2] + p_{1}x[ 1]    \\
    \cdots &= \cdots    \nonumber
\end{align}
and so on and so forth. Each step of the calculation requires the knowledge of the previously calculated value of the output sample (delayed value of the output), the present value of the input sample and the previous one (delayed value of the input).
As a result, the first-order difference equation can be interpreted as a valid computational algorithm.


\medskip
\marginpar{Basic building blocks}
The computational algorithm of an LTI digital filter can be conveniently represented in \textbf{block diagram form} using the \textbf{basic building blocks} showed in Figure \ref{fig:L21_S20_1}.

\begin{figure}[!h]
    \centering
    \includegraphics[width=0.75\textwidth]{\figpath{21}/21_images/S20_1.pdf}
    \caption{\label{fig:L21_S20_1} Basic building blocks of a block diagram.}
\end{figure}

There are several advantages when adopting this representation. In fact, it is:
\marginpar{Advantages of block diagram representation}
\begin{itemize}
    \item easy to write down the computational algorithm by inspection;
    \item easy to analyze the block diagram to determine the explicit relation between output and input;
    \item easy to manipulate a block diagram to derive other ``equivalent'' block diagrams yielding different computational algorithms;
    \item easy to determine the hardware requirements;
    \item easier to develop block diagram representations from the transfer function directly.
\end{itemize}

\medskip
\marginpar{Analysis block diagrams}
The \textbf{analysis of block diagrams} is carried out by writing down the expressions for the output signals of each adder as a sum of its input signals, and developing a set of equations relating the fiter input and output signals in terms of all internal signals. Eliminating the unwanted internal variables, then, results in the expression for the output signal as a function of the input signal and the filter parameters, that are the multiplier coefficients.

\begin{example}{Analysis of block diagrams}{}
    \marginnote{\flushleft\textsl{\small Example with a single-loop feedback structure}}
    We consider the \textbf{single-loop feedback structure} showed below.

    \begin{center}
        \includegraphics[width=0.3\textwidth]{\figpath{21}/21_images/S23_1.pdf}
    \end{center}

    The output \( E(z) \) of the adder is:
    \begin{equation}
        E(z)
        =
        X(z) + G_{2}(z)Y(z)
        \label{eq:L21_S23_1}
    \end{equation}
    However, from the scheme we can see that:
    \begin{equation}
        Y(z)
        =
        G_{1}(z)E(z)
        \label{eq:L21_S23_2}
    \end{equation}
    Eliminating \( E(z) \) from Eqs. \ref{eq:L21_S23_1} and \ref{eq:L21_S23_2}, we arrive at:
    \begin{equation}
        \qty[1 - G_{1}(z)G_{2}(z)] Y(z)
        =
        G_{1}(z)X(z)
        \label{eq:L21_S24_1}
    \end{equation}
    which leads to:
    \begin{equation}
        H(z)
        =
        \frac{Y(z)}{X(z)}
        =
        \frac{G_{1}(z)}{1 - G_{1}(z)G_{2}(z)}
        \label{eq:L21_S24_2}
    \end{equation}
\end{example}

\begin{example}{Analysis of block diagrams}{}
    \marginnote{\flushleft\textsl{\small Example with a cascaded lattice structure}}
    We analyze the \textbf{cascaded lattice structure} showed in the scheme below, where the \( z \)-dependence of signal variables is not showed for brevity.

    \begin{center}
        \includegraphics[width=0.5\textwidth]{\figpath{21}/21_images/S25_1.pdf}
    \end{center}

    The output signals of the four adders are given by:
    \begin{align}
        W_{1} &= X - \alpha S_{2}           \\
        W_{2} &= W_{1} - \delta S_{1}       \\
        W_{3} &= S_{1} + \varepsilon W_{2}  \\
        Y     &= \beta W_{1} + \gamma S_{2}
    \end{align}
    Moreover, from the scheme we can observe:
    \begin{align}
        S_{2} &= z^{-1} W_{3}   \\
        S_{1} &= z^{-1} W_{2}
    \end{align}
    Substituting the last two relations in the first four equations, we get:
    \begin{align}
        W_{1} &= X - \alpha z^{-1} W_{3}            \\
        W_{2} &= W_{1} - \delta z^{-1} W_{2}        \\
        W_{3} &= z^{-1} W_{2} + \varepsilon W_{2}   \\
        Y     &= \beta W_{1} + \gamma z^{-1} W_{3}
    \end{align}
    Therefore, we get:
    \begin{align}
        W_{2} &= \frac{W_{1}}{1 + \delta z^{-1}}    \\
        W_{3} &= \qty(\varepsilon + z^{-1}) W_{2}
    \end{align}
    Combining the last two equations we get:
    \begin{equation}
        W_{3}
        =
        \frac{\varepsilon + z^{-1}}{1 + \delta z^{-1}} W_{1}
        \label{eq:L21_S28_1}
    \end{equation}
    Substituting Eq. \ref{eq:L21_S28_1} the ones for \( W_{1} \) and \( Y \), we finally arrive at:
    \begin{equation}
        H(z)
        =
        \frac{Y}{X}
        =
        \frac{\beta + \qty(\beta\delta + \gamma\varepsilon)z^{-1} + \gamma z^{-2}}{1 + \qty(\delta + \alpha\varepsilon)z^{-1} + \alpha z^{-2}}
        \label{eq:L21_S28_2}
    \end{equation}
\end{example}



\subsection{The delay-free loop problem}
\marginpar{Necessary condition for physical realizability of the structure}
For the physical realizability of the digital filter structure, it is necessary that the block representation contains no delay-free loops. To illustrate the \textbf{delay-free loop problem}, we consider the structure in Figure \ref{fig:L21_S29_1}.

\begin{figure}[!h]
    \centering
    \includegraphics[width=0.4\textwidth]{\figpath{21}/21_images/S29_1.pdf}
    \caption{\label{fig:L21_S29_1} Example of structure with a delay loop.}
\end{figure}

From the analysis of this structure, we get:
\begin{align}
    u[n] &= w[n] + y[n]         \label{eq:L21_S30_1}\\
    y[n] &= B\qty(v[n] + Au[n]) \label{eq:L21_S30_2}
\end{align}
which, when combined, results in:
\begin{equation}
    y[n]
    =
    B\qty(v[n] + A\qty(w[n] + y[n]))
    \label{eq:L21_S30_3}
\end{equation}
The determination of the current value of \( y[n] \) requires the knowledge of the same value.
However, this is physically impossible to achieve due to the finite time required to carry out all arithmetic operations on a digital machine.

There exist methods to detect \marginpar{Solving the delay-free loop problem}the presence of delay-free loops in an arbitrary structure, along with methods to locate and remove these loops without the overall input-output relation.
One possibility of removal is achieved by replacing the portion of the overall structure containing the delay-free loops by an equivalent realization with no delay-free loops.
To illustrate the process, let us consider again Eqs. \ref{eq:L21_S30_1} and \ref{eq:L21_S30_2}, characterizing the structure in Figure \ref{fig:L21_S29_1}.
Substituing Eq. \ref{eq:L21_S30_2} into Eq. \ref{eq:L21_S30_1}, we arrive at:
\begin{equation}
    u[n]
    =
    w[n] + Bv[n] + ABu[n]
    \label{eq:L21_S33_1}
\end{equation}
Solving Eq. \ref{eq:L21_S33_1}, we get:
\begin{equation}
    u[n]
    =
    \frac{1}{1 - AB} \qty(w[n] + Bv[n])
    \label{eq:L21_S33_2}
\end{equation}
By reordering Eq. \ref{eq:L21_S30_1}, we get:
\begin{equation}
    y[n]
    =
    u[n] - w[n]
    \label{eq:L21_S33_3}
\end{equation}
Therefore, we have obtained a delay-free loop realization based on Eqs. \ref{eq:L21_S33_2} and \ref{eq:L21_S33_3}, and the scheme of the new structure is showed in Figure \ref{fig:L21_S34_1}.

\begin{figure}[!h]
    \centering
    \includegraphics[width=0.4\textwidth]{\figpath{21}/21_images/S34_1.pdf}
    \caption{\label{fig:L21_S34_1} Structure of Figure \ref{fig:L21_S29_1} after removal of delay-free loops.}
\end{figure}



\subsection{Canonic and non-canonic structures}
\marginpar{Canonic and non-canonic structures}
A digital filter structure is said to be \textbf{canonic} if the number of delays in the block diagram representation is equal to the order of the transfer function. Otherwise, it is a \textbf{non-canonic structure}.

\begin{example}{Canonic and non-canonic structures}{}
    The structure characterized by:
    \begin{equation}
        y[n]
        =
        -d_{1}y[n-1] + p_{0}x[n] + p_{1}x[n-1]
        \label{eq:L21_S36_1}
    \end{equation}
    and showed in the scheme below is non-canonic, as it employs two delays to realize a first-order difference equation.

    \begin{center}
        \includegraphics[width=0.4\textwidth]{\figpath{21}/21_images/S36_1.pdf}
    \end{center}
\end{example}

\end{document}

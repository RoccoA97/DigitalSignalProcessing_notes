\providecommand{\main}{../../main}
\providecommand{\figpath}[1]{\main/../lessons/#1}
\documentclass[../../main/main.tex]{subfiles}



\begin{document}

\newdate{date}{05}{11}{2020}
\marginpar{ \textbf{Lecture 12.} \\  \displaydate{date}.}

\section{Discrete Fourier Transform}
We have discussed the DTFT for a discrete-time function given by:
\begin{equation}
    X(e^{j\omega})
    =
    \sum_{n=-\infty}^{\infty} x[n]e^{-j\omega n}
    \label{eq:L12_S03_1}
\end{equation}
and the inverse DTFT:
\begin{equation}
    x[n]
    =
    \frac{1}{2\pi} \int_{-\pi}^{\pi} X(e^{j\omega}) e^{j\omega n} \d{\omega}
    \label{eq:L12_S03_2}
\end{equation}
\marginpar{Drawbacks of DTFT}The pair and their properties and applications have some limitations. The input signal is usually aperiodic and may be finite in length.

\begin{figure}[!h]
    \centering
    \includegraphics[width=0.5\textwidth]{\figpath{12}/12_images/S04_1.pdf}
    \caption{\label{fig:L12_S04_1} In order from top to bottom, a finite-length signal, its magnitude spectrum, its phase spectrum.}
\end{figure}

Moreover, we often do not have an infinite amount of data which is required by DTFT. For example in a computer we cannot calculate uncountable infinite (continuum) of frequencies as required by DTFT.
\marginpar{Discrete Fourier transform as solution}
Thus, we use DTF to look at finite segment of data. We only observe the data through a window:
\begin{align}
    x_{0}[n] &= x[n]w_{R}[n]    \\
    w_{R}[n] &=
    \begin{cases}
        1   &   n=0,1,\dots,N-1 \\
        0   &   \text{otherwise}
    \end{cases}
\end{align}
In this case, the \( x_{0}[n] \) is just a sampled data between \( n=0 \), \( n=N-1 \) (so, \( N \) points).

The solution to our problems is given by the Discrete Fourier Transform (DFT).

\marginpar{Definition of discrete Fourier transform}
\begin{definition}{Discrete Fourier Transform (DFT)}{}
    The simplest relation between a length-\( N \) sequence \( x[n] \), defined for \( 0 \le n \le N-1 \), and its DTFT \( X(e^{j\omega}) \) is obtained by uniformly sampling on the \( \omega \)-axis between \( 0 \le \omega \le 2\pi \) at \( \omega_{k} = \frac{2\pi k}{N} \), for \( 0 \le k \le N-1 \). From the definition of the DTFT we thus have:
    \begin{equation}
        X[k]
        =
        \qty[X(e^{j\omega})]_{\omega=\frac{2\pi k}{N}}
        =
        \sum_{n=0}^{N-1} x[n]e^{-j 2\pi k \frac{n}{N}}
        \label{eq:L12_S06_1}
    \end{equation}
    Note that \( X[k] \) is also a length-\( N \) sequence in the frequency domain and it is called the \textbf{Discrete Fourier Transform (DFT)} of the sequence \( x[n] \). Using the notation \( W_{N} = e^{-j \frac{2\pi}{N}} \), the DFT is usually expressed as:
    \begin{equation}
        X[k]
        =
        \sum_{n=0}^{N-1} x[n] W_{N}^{kn},
        \qquad
        0 \le k \le N-1
        \label{eq:L12_S07_1}
    \end{equation}
\end{definition}

\marginpar{Definition of inverse discrete Fourier transform}
\begin{definition}{Inverse Discrete Fourier Transform (IDFT)}{}
    The \textbf{Inverse Discrete Fourier Transform (IDFT)} is given by:
    \begin{equation}
        x[n]
        =
        \frac{1}{N} \sum_{k=0}^{N-1} X[k] W_{N}^{-kn},
        \qquad
        0 \le n \le N-1
        \label{eq:L12_S08_1}
    \end{equation}
\end{definition}
\marginpar{Validation of the DFT definition}
To verify the above expression, we multiply both sides of the above equation by \( W_{N}^{\ell n} \) and sum the result from \( n=0 \) to \( n=N-1 \), resulting in:
\begin{align}
    \sum_{n=0}^{N-1} x[n] W_{N}^{\ell n}
    &=
        \sum_{n=0}^{N-1} \qty(\frac{1}{N} \sum_{k=0}^{N-1} X[k] W_{N}^{-kn}) W_{N}^{\ell n} \nonumber   \\
    &=
        \frac{1}{N} \sum_{n=0}^{N-1} \sum_{k=0}^{N-1} X[k] W_{N}^{-(k-\ell)n}   \nonumber   \\
    &=
        \frac{1}{N} \sum_{k=0}^{N-1} \sum_{n=0}^{N-1} X[k] W_{N}^{-(k-\ell)n}
\end{align}
Making use of the identity:
\begin{equation}
    \sum_{n=0}^{N-1} W_{N}^{-(k-\ell)n}
    =
    \begin{cases}
        N   &   k-\ell = rN, \ r \in \mathbb{Z} \\
        0   &   \text{otherwise}
    \end{cases}
    \label{eq:L12_S10_1}
\end{equation}
we observe that the RHS of the last equation is equal to \( X[\ell] \). Hence:
\begin{equation}
    \sum_{n=0}^{N-1} x[n] W_{N}^{\ell n}
    =
    X[\ell]
    \label{eq:L12_S10_2}
\end{equation}

Now, we focus on some examples to understand the practical way to compute the DFT.

\marginpar{Examples of discrete Fourier transform calculation}
\begin{example}{Discrete Fourier Transform}{}
    Consider the length-\( N \) sequence:
    \begin{equation}
        x[n]
        =
        \begin{cases}
            1   &   n=0 \\
            0   &   1 \le n \le N-1
        \end{cases}
        \label{eq:L12_S11_1}
    \end{equation}
    Its \( N \)-point DFT is given by:
    \begin{equation}
        X[k]
        =
        \sum_{n=0}^{N-1} x[n] W_{N}^{kn}
        =
        x[0] W_{N}^{0}
        =
        1
        \label{eq:L12_S11_2}
    \end{equation}
    with \( 0 \le k \le N-1 \).
\end{example}

\begin{example}{Discrete Fourier Transform}{}
    Consider the length-\( N \) sequence:
    \begin{equation}
        y[n]
        =
        \begin{cases}
            1   &   n=m \\
            0   &   0 \le n \le m-1, \ m+1 \le n \le N-1
        \end{cases}
        \label{eq:L12_S12_1}
    \end{equation}
    Its \( N \)-point DFT is given by:
    \begin{equation}
        Y[k]
        =
        \sum_{n=0}^{N-1} y[n] W_{N}^{kn}
        =
        y[m] W_{N}^{km}
        =
        W_{N}^{km}
        \label{eq:L12_S12_2}
    \end{equation}
    with \( 0 \le k \le N-1 \).
\end{example}

\begin{example}{Discrete Fourier Transform}{}
    Consider the length-\( N \) sequence defined for \( 0 \le n \le N-1 \):
    \begin{equation}
        g[n]
        =
        \cos\qty(\frac{2\pi rn}{N}),
        \qquad
        0 \le r \le N-1
        \label{eq:L12_S13_1}
    \end{equation}
    Using trigonometic identities, we can rewrite:
    \begin{equation}
        g[n]
        =
        \frac{1}{2} \qty(e^{j2\pi r \frac{n}{M}} + e^{-j2\pi r \frac{n}{N}})
        =
        \frac{1}{2} \qty(W_{N}^{-rn} + W_{N}^{rn})
        \label{eq:L12_S13_2}
    \end{equation}
    The \( N \)-point DFT of \( g[n] \) is thus given by:
    \begin{equation}
        G[k]
        =
        \sum_{n=0}^{N-1} g[n] W_{N}^{kn}
        =
        \frac{1}{2} \qty(\sum_{n=0}^{N-1} W_{N}^{-(r-k)n} + \sum_{n=0}^{N-1} W_{N}^{(r+k)n} )
        \label{eq:L12_S14_1}
    \end{equation}
    with \( 0 \le k \le N-1 \). Making use of the identity:
    \begin{equation}
        \sum_{n=0}^{N-1} W_{N}^{-(k-\ell)n}
        =
        \begin{cases}
            N   &   k-\ell = rN, \ r \in \mathbb{Z} \\
            0   &   \text{otherwise}
        \end{cases}
        \label{eq:L12_S15_1}
    \end{equation}
    we get:
    \begin{equation}
        G[k]
        =
        \begin{cases}
            \frac{N}{2} &   k=r \\
            \frac{N}{2} &   k=N-r   \\
            0   &   \text{otherwise}
        \end{cases}
        \label{eq:L12_S15_2}
    \end{equation}
    with \( 0 \le k \le N-1 \).
\end{example}



\subsection{Matrix relations}
\marginpar{DFT matrix form}
The DFT analysis and synthesis formulas can now be easily expressed in a familiar matrix notation. More precisely, the DFT samples previously defined in Eq. \ref{eq:L12_S07_1} can be expressed in matrix form as:
\begin{equation}
    \mathbf{X}
    =
    \mathbf{D}_{N} \mathbf{x}
    \label{eq:L12_S16_2}
\end{equation}
where:
\begin{equation}
    \mathbf{X}
    =
        \begin{bmatrix}
            X[0]    \\
            X[1]    \\
            \vdots  \\
            X[N-1]
        \end{bmatrix},
    \qquad
    \mathbf{x}
    =
        \begin{bmatrix}
            x[0]    \\
            x[1]    \\
            \vdots  \\
            x[N-1]
        \end{bmatrix}
\end{equation}
\marginpar{DFT matrix}and \( \mathbf{D}_{N} \) is the \( N \times N \) \textbf{DFT matrix} given by:
\begin{equation}
    \mathbf{D}_{N}
    =
    \begin{bmatrix}
        1   &   1   &   1   &   \cdots & 1   \\
        1   &   W_{N}^{1}   &   W_{N}^{2}   &   \cdots & W_{N}^{(N-1)}   \\
        1   &   W_{N}^{2}   &   W_{N}^{4}   &   \cdots & W_{N}^{2(N-1)}   \\
        \vdots  &   \vdots  &   \vdots  &   \ddots  &   \vdots  \\
        1   &   W_{N}^{(N-1)}   &   W_{N}^{2(N-1)}   &   \cdots & W_{N}^{(N-1)^2}   \\
    \end{bmatrix}
    \label{eq:L12_S17_1}
\end{equation}
Likewise, the IDFT relation given by:
\begin{equation}
    x[n]
    =
    \frac{1}{N} \sum_{k=0}^{N-1} X[k] W_{N}^{-kn},
    \qquad
    0 \le n \le N-1
    \label{eq:L12_S18_1}
\end{equation}
can be expressed in matrix form as:
\begin{equation}
    \mathbf{x}
    =
    \mathbf{D}_{N}^{-1} \mathbf{X}
    \label{eq:L12_S18_2}
\end{equation}
where \( \mathbf{D}_{N}^{-1} \) is the \( N \times N \) \textbf{inverse DFT matrix}, given by:
\marginpar{Inverse DFT matrix}
\begin{equation}
    \mathbf{D}_{N}^{-1}
    =
    \begin{bmatrix}
        1   &   1   &   1   &   \cdots & 1   \\
        1   &   W_{N}^{-1}   &   W_{N}^{-2}   &   \cdots & W_{N}^{-(N-1)}   \\
        1   &   W_{N}^{-2}   &   W_{N}^{-4}   &   \cdots & W_{N}^{-2(N-1)}   \\
        \vdots  &   \vdots  &   \vdots  &   \ddots  &   \vdots  \\
        1   &   W_{N}^{-(N-1)}   &   W_{N}^{-2(N-1)}   &   \cdots & W_{N}^{-(N-1)^2}   \\
    \end{bmatrix}
    =
    \frac{1}{N} \mathbf{D}_{N}^{*}
    \label{eq:L12_S19_1}
\end{equation}



\subsection{DTFT from DFT by interpolation and sampling of DTFT}
\marginpar{DTFT from DFT interpolation}
The \( N \)-point DFT \( X[k] \) of a length-\( N \) sequence \( x[n] \) is simply the frequency samples of its DTFT \( X(e^{j\omega}) \) evaluated at \( N \) uniformly spaced frequency points:
\begin{equation}
    \omega
    =
    \omega_{k}
    =
    \frac{2\pi k}{N},
    \qquad
    0 \le k \le N-1
    \label{eq:L12_S20_1}
\end{equation}
Given the \( N \)-point DFT \( X[k] \) of a length-\( N \) sequence \( x[n] \), its DTFT \( X(e^{j\omega}) \) can be uniquely determined from \( X[k] \). Thus:
\begin{align}
    X(e^{j\omega})
    &=
        \sum_{n=0}^{N-1} x[n] e^{-j\omega n}    \nonumber   \\
    &=
        \sum_{n=0}^{N-1} \qty[\frac{1}{N} \sum_{k=0}^{N-1} X[k]W_{N}^{-kn}] e^{-j\omega n}  \nonumber   \\
    &=
        \frac{1}{N} \sum_{k=0}^{N-1} X[k] \underbrace{\sum_{n=0}^{N-1} e^{-j\qty(\omega - \frac{2\pi k}{N})n}}_{S}
\end{align}

To develop a compact expression for the sum \( S \), let \( r = e^{-j\qty(\omega - \frac{2\pi k}{N})} \). Then, we can rewrite the sum as:
\begin{equation}
    S
    =
    \sum_{n=0}^{N-1} r^{n}
    \label{eq:L12_S22_1}
\end{equation}
From Eq. \ref{eq:L12_S22_1}:
\begin{align}
    rS
    &=
        \sum_{n=1}^{N} r^{n} = 1 + \sum_{n=1}^{N-1} r^{n} r^{N} - 1 \nonumber   \\
    &=
        \sum_{n=0}^{N-1} r^{n} + r^{N} - 1 = S + r^{N} - 1
\end{align}
or, equivalently:
\begin{equation}
    S - rS
    =
    (1-r)S
    =
    1 - r^{N}
    \label{eq:L12_S23_1}
\end{equation}
Hence:
\begin{equation}
    S
    =
    \frac{1-r^{N}}{1-r}
    =
    \frac{1-e^{-j\qty(\omega n - 2\pi k)}}{1-e^{-j\qty(\omega - \frac{2\pi k}{N})}}
    =
    \frac{\displaystyle\sin\qty(\frac{\omega N  2\pi k}{2})}{\displaystyle\sin\qty(\frac{\omega N - 2\pi k}{2N})} e^{-j \qty(\frac{\omega - 2\pi k}{N}) \qty(\frac{N-1}{2})}
\end{equation}
Therefore:
\begin{equation}
    X(e^{j\omega})
    =
    \frac{1}{N} \sum_{k=0}^{N-1} X[k]
    \frac{\displaystyle\sin\qty(\frac{\omega N  2\pi k}{2})}{\displaystyle\sin\qty(\frac{\omega N - 2\pi k}{2N})}
    e^{-j \qty(\frac{\omega - 2\pi k}{N}) \qty(\frac{N-1}{2})}
    \label{eq:L12_S24_1}
\end{equation}


\medskip
\marginpar{Sampling the DTFT}
Now, we study the inverse operation. We consider a sequence \( x[n] \) with a DTFT \( X(e^{j\omega}) \). We sample \( X(e^{j\omega}) \) at \( N \) equally spaced points \( \omega_{k} = \frac{2\pi k}{N} \), \( 0 \le k \le N-1 \), developing the \( N \) frequency samples \( \qty{X(e^{j\omega_{k}})} \).
These \( N \) frequency samples can be considered as an \( N \)-point DFT \( Y[k] \) whose \( N \)-point inverse DFT is a length-\( N \) sequence \( y[n] \). Now:
\begin{equation}
    X(e^{j\omega})
    =
    \sum_{\ell=-\infty}^{\infty} x[\ell] e^{-j\omega \ell}
    \label{eq:L12_S26_1}
\end{equation}
Thus:
\begin{equation}
    Y[k]
    =
    X(e^{j\omega_{k}})
    =
    X(e^{j\frac{2\pi k}{N}})
    =
    \sum_{\ell=-\infty}^{\infty} x[\ell] e^{-j 2\pi k \frac{\ell}{N}}
    =
    \sum_{\ell=-\infty}^{\infty} x[\ell] W_{N}^{k\ell}
    \label{eq:L12_S26_2}
\end{equation}
The inverse DFT of \( Y[k] \) yields:
\begin{align}
    y[n]
    &=
        \frac{1}{N} \sum_{k=0}^{N-1} Y[k] W_{N}^{-kn}   \nonumber   \\
    &=
        \frac{1}{N} \sum_{k=0}^{N-1} \sum_{\ell=-\infty}^{\infty} x[\ell] W_{N}^{k\ell} W_{N}^{-kn} \nonumber   \\
    &=
        \sum_{\ell=-\infty}^{\infty} x[\ell] \qty[\sum_{k=0}^{N-1} W_{N}^{-k(n-\ell)}]  \nonumber   \\
    &=
        \sum_{m=-\infty}^{\infty} x[n+mN]
\end{align}
with \( 0 \le n \le N-1 \), where in the last passage the identity in Eq. \ref{eq:L12_S15_1} is employed. Thus, \( y[n] \) is obtained from \( x[n] \) by adding an infinite number of shifted replicas of \( x[n] \), with each replica shifted by an integer multiple of \( N \) sampling instants, and observing the sum only for the interval \( 0 \le n \le N-1 \).

To apply the last result to finite-length sequences, we assume that the samples outside the specified range are zeros. Thus, if \( x[n] \) is a length-\( M \) sequence with \( M \le N \), then \( y[n] = x[n] \) for \( 0 \le n \le N-1 \).
If \( M > N \), there is a time-domain aliasing of samples of \( x[n] \) in generating \( y[n] \), and \( x[n] \) cannot be recovered from \( y[n] \).

\begin{example}{Aliasing}{}
    Let \( x[n] = \qty{0,1,2,3,4,5} \).
    \marginnote{\flushleft\textsl{\small Aliasing}}
    By sampling its DTFT \( X(e^{j\omega}) \) at \( \omega_{k} = \frac{2\pi k}{4} \), with \( 0 \le k \le 3 \), and then applying a 4-point IDFT to these samples, we arrive at the sequence \( y[n] \) given by:
    \begin{equation}
        y[n]
        =
        x[n] + x[n+4] + x[n-4],
        \qquad
        0 \le n \le 3
        \label{eq:L12_S30_2}
    \end{equation}
    We get \( y[n] = \qty{4,6,2,3} \). However, \( x[n] \) cannot be recovered from \( y[n] \).
\end{example}



\subsection{DFT properties}
\marginpar{DFT properties}
Like the DTFT, the DFT also satisfies a number of properties that are useful in signal processing applications. Some of these properties are essentially identical to those of the DTFT, while some others are somewhat different. A summary of the DFT properties are given in Figures \ref{fig:L12_S32_1}, \ref{fig:L12_S33_1} and \ref{fig:L12_S34_1}.

\begin{figure}[!h]
    \centering
    \includegraphics[width=0.5\textwidth]{\figpath{12}/12_images/S32_1.pdf}
    \caption{\label{fig:L12_S32_1} Symmetry relations of DFT for a complex sequence \( x[n] \).}
\end{figure}

\begin{figure}[!h]
    \centering
    \includegraphics[width=0.5\textwidth]{\figpath{12}/12_images/S33_1.pdf}
    \caption{\label{fig:L12_S33_1} Symmetry relations of DFT for a real sequence \( x[n] \).}
\end{figure}

\begin{figure}[!h]
    \centering
    \includegraphics[width=0.5\textwidth]{\figpath{12}/12_images/S34_1.pdf}
    \caption{\label{fig:L12_S34_1} General properties of DFT.}
\end{figure}



\subsection{Circular shift and convolution of a sequence}
After having listed some of the most common and useful properties of DFT, we focus on the operations of circular shift and circular convolution. Let us start from the former.

\medskip
\marginpar{Circular shift of a sequence}
This property is analogous to the time-shifting property of the DTFT but with a difference. We inspect it by considering length-\( N \) sequences defined for \( 0 \le n \le N-1 \).
The sample values of such sequences are equal to zero for values of \( n < 0 \) and \( n \ge N \).
If \( x[n] \) is such a sequence, then for any arbitrary integer \( n_{0} \) the shifted sequence \( x_{1}[n] = x[n-n_{0}] \) is no longer defined for the range \( 0 \le n \le N-1 \). We thus need to define another type of shift that will always keep the shifted sequence in the range \( 0 \le n \le N-1 \).
\marginpar{Circular shift and modulo operation}
The desired shift, called the \textbf{circular shift}, is defined using a \textbf{modulo operation}:
\begin{equation}
    x_{c}[n]
    =
    x[\expval{n-n_{0}}_{N}]
    \label{eq:L12_S37_1}
\end{equation}
For \( n_{0} > 0 \) (\textbf{right circular shift}), the above equation implies:
\begin{equation}
    x_{c}[n]
    =
    \begin{cases}
        x[n-n_{0}]  &   n_{0} \le n \le N-1 \\
        x[N-n_{0}+n]&   0 \le n \le n_{0}
    \end{cases}
    \label{eq:L12_S37_2}
\end{equation}

An illustration of the concept of circular shift is showed in Figure \ref{fig:L12_S38_1}. As it is possible to observe, a right circular shift by \( n_{0} \) is equivalent to a left circular shift by \( N - n_{0} \) sample periods. A circular shift by an integer number \( n_{0} \) greater than \( N \) is equivalent to a circular shift by \( \expval{n_{0}}_{N} \).

\begin{figure}[!h]
    \centering
    \includegraphics[width=0.75\textwidth]{\figpath{12}/12_images/S38_1.pdf}
    \caption{\label{fig:L12_S38_1} Illustration of circular shift.}
\end{figure}



\medskip
\marginpar{Circular convolution of a sequence}
Now, we pass to the circular convolution. This operation is analogous to the linear convolution, but with a difference. Consider two length-\( N \) sequences, \( g[n] \) and \( h[n] \), respectively. Their linear convolution results in a length-\( (2N-1) \) sequence \( y_{L}[n] \) given by:
\begin{equation}
    y_{L}[n]
    =
    \sum_{m=0}^{N-1} g[m]h[n-m],
    \qquad
    0 \le n \le 2N-2
    \label{eq:L12_S40_1}
\end{equation}
In computing \( y_{L}[n] \), we have assumed that both length-\( N \) sequences have been zero-padded to extend their lengths to \( 2N - 1 \).
The longer form of \( y_{L}[n] \) results from the time-reversal of the sequence \( h[n] \) and its linear shift to the right.
The first nonzero value of \( y_{L}[n] \) is \( y_{L}[0] = g[0]h[0] \) and the last nonzero value is \( y_{L}[2N-2] = g[N-1]h[N-1] \).

To develop a convolution-like operation resulting in a length-\( N \) sequence \( y_{C}[n] \), we need to define a \textbf{circular time-reversal}, and then apply a \textbf{circular time-shift}.
\marginpar{Circular time-reversal and time-shift for circular convolution}
Resulting operation, called a \textbf{circular convolution}, is defined by:
\begin{equation}
    y_{C}[n]
    =
    \sum_{m=0}^{N-1} g[m]h[\expval{n-m}_{N}],
    \qquad
    0 \le n \le N-1
    \label{eq:L12_S42_1}
\end{equation}
Since the operation defined involves two length-\( N \) sequences, it is often referred to as an \textbf{\( N \)-point circular convolution}, denoted as:
\begin{equation}
    y[n]
    =
    g[n] \circled{N} h[n]
    \label{eq:L12_S43_1}
\end{equation}
We remark that the circular convolution is commitative, namely:
\begin{equation}
    g[n] \circled{N} h[n]
    =
    h[n] \circled{N} g[n]
    \label{eq:L12_S43_2}
\end{equation}
Now, we study an example to understand how this operation works.

\begin{example}{Circular convolution}{}
    Here, we compute the 4-point circular convolution of the two length-\( 4 \) sequences:
    \begin{align}
        g[n] &= \qty{{\color{red}1}, 2, 0, 1}   \\
        h[n] &= \qty{{\color{red}2}, 2, 1, 1}
    \end{align}
    as sketched in the figure below.

    \begin{center}
        \includegraphics[width=0.5\textwidth]{\figpath{12}/12_images/S44_1.pdf}
    \end{center}

    The result is a length-\( 4 \) sequence \( y_{C}[n] \) given by:
    \begin{equation}
        y_{C}[n]
        =
        g[n] \circled{4} h[n]
        =
        \sum_{m=0}^{3} g[m] h[\expval{n-m}_{4}],
        \qquad
        0 \le n \le 3
        \label{eq:L12_S45_1}
    \end{equation}
    We can observe that:
    \begin{align}
        y_{C}[0]
        &=
            \sum_{m=0}^{3} g[m] h[\expval{-m}_{4}]  \nonumber   \\
        &=
            g[0]h[0] + g[1]h[3] + g[2]h[2] + g[3]h[1]   \nonumber   \\
        &=
            (1 \cdot 2) + (2 \cdot 1) + (0 \cdot 1) + (1 \cdot 2) = 6
    \end{align}
    \begin{align}
        y_{C}[1]
        &=
            \sum_{m=0}^{3} g[m] h[\expval{1-m}_{4}]  \nonumber   \\
        &=
            g[0]h[1] + g[1]h[0] + g[2]h[3] + g[3]h[2]   \nonumber   \\
        &=
            (1 \cdot 2) + (2 \cdot 2) + (0 \cdot 1) + (1 \cdot 1) = 7
    \end{align}
    \begin{align}
        y_{C}[2]
        &=
            \sum_{m=0}^{3} g[m] h[\expval{2-m}_{4}]  \nonumber   \\
        &=
            g[0]h[2] + g[1]h[1] + g[2]h[0] + g[3]h[3]   \nonumber   \\
        &=
            (1 \cdot 1) + (2 \cdot 2) + (0 \cdot 2) + (1 \cdot 1) = 6
    \end{align}
    \begin{align}
        y_{C}[3]
        &=
            \sum_{m=0}^{3} g[m] h[\expval{3-m}_{4}]  \nonumber   \\
        &=
            g[0]h[3] + g[1]h[2] + g[2]h[1] + g[3]h[0]   \nonumber   \\
        &=
            (1 \cdot 1) + (2 \cdot 1) + (0 \cdot 2) + (1 \cdot 2) = 5
    \end{align}

    \begin{center}
        \includegraphics[width=0.3\textwidth]{\figpath{12}/12_images/S47_1.pdf}
    \end{center}
\end{example}

\begin{example}{Circular convolution}{}
    Here, we compute the 4-point circular convolution of the two length-\( 4 \) sequences in the figure below:

    \begin{center}
        \includegraphics[width=0.6\textwidth]{\figpath{12}/12_images/S48_1.pdf}
    \end{center}

    The \( 4 \)-point DFT \( G[k] \) of \( g[n] \) is given by:
    \begin{align}
        G[k]
        &=
            g[0] + g[1]e^{-j\frac{2\pi k}{4}} + g[2]e^{-j\frac{4\pi k}{4}} + g[3]e^{-j\frac{6\pi k}{4}} \nonumber   \\
        &=
            1 + 2e^{-j \frac{\pi k}{2}} + e^{-j \frac{3\pi k}{2}},
            \qquad\qquad\qquad\qquad\qquad\qquad
            0 \le k \le 3
    \end{align}
    Therefore:
    \begin{align}
        G[0] &= 1 + 2 + 1 = 4   \\
        G[1] &= 1 -j2 + j = 1 - j   \\
        G[2] &= 1 - 2 - 1 = -2  \\
        G[3] &= 1 +j2 - j = 1 + j
    \end{align}
    Likewise:
    \begin{align}
        H[k]
        &=
            h[0] + h[1]e^{-j\frac{2\pi k}{4}} + h[2]e^{-j\frac{4\pi k}{4}} + h[3]e^{-j\frac{6\pi k}{4}} \nonumber   \\
        &=
            2 + 2e^{-j \frac{\pi k}{2}} + e^{-j \pi k} + e^{-j \frac{3\pi k}{2}},
            \qquad\qquad\qquad\qquad
            0 \le k \le 3
    \end{align}
    Hence, we get:
    \begin{align}
        H[0] &= 2 + 2 + 1 + 1 = 6   \\
        H[1] &= 2 -j2 - 1 + j = 1 - j   \\
        H[2] &= 2 - 2 + 1 - 1 = 0   \\
        H[3] &= 2 +j2 - 1 - j = 1 + j
    \end{align}
    The two \( 4 \)-pont DFTs can also be computed using the matrix relations:
    \begin{align}
        \begin{bmatrix}
            G[0]    \\
            G[1]    \\
            G[2]    \\
            G[3]
        \end{bmatrix}
        &=
            \mathbf{D}_{4}
            \begin{bmatrix}
                g[0]    \\
                g[1]    \\
                g[2]    \\
                g[3]
            \end{bmatrix}
            =
            \begin{bmatrix}
                1   &   1   &   1   &   1   \\
                1   &   -j  &   -1  &   j   \\
                1   &   -1  &   1   &   -1  \\
                1   &   j   &   -1  &   -j
            \end{bmatrix}
            \begin{bmatrix}
                1   \\
                2   \\
                0   \\
                1
            \end{bmatrix}
            =
            \begin{bmatrix}
                4   \\
                1-j \\
                -2  \\
                1+j
            \end{bmatrix}   \\
        \begin{bmatrix}
            H[0]    \\
            H[1]    \\
            H[2]    \\
            H[3]
        \end{bmatrix}
        &=
            \mathbf{D}_{4}
            \begin{bmatrix}
                h[0]    \\
                h[1]    \\
                h[2]    \\
                h[3]
            \end{bmatrix}
            =
            \begin{bmatrix}
                1   &   1   &   1   &   1   \\
                1   &   -j  &   -1  &   j   \\
                1   &   -1  &   1   &   -1  \\
                1   &   j   &   -1  &   -j
            \end{bmatrix}
            \begin{bmatrix}
                2   \\
                2   \\
                1   \\
                1
            \end{bmatrix}
            =
            \begin{bmatrix}
                6   \\
                1-j \\
                0   \\
                1+j
            \end{bmatrix}
    \end{align}
    where \( \mathbf{D}_{4} \) is the \( 4 \)-point DFT matrix. If \( Y_{C}[k] \) denotes the \( 4 \)-point DFT of \( y_{C}[n] \), we observe \( Y_{C}[k] = G[k]H[k] \) for \( 0 \le k \le 3 \), so:
    \begin{equation}
        \begin{bmatrix}
            Y_{C}[0]    \\
            Y_{C}[1]    \\
            Y_{C}[2]    \\
            Y_{C}[3]
        \end{bmatrix}
        =
        \begin{bmatrix}
            G[0]H[0]    \\
            G[1]H[1]    \\
            G[2]H[2]    \\
            G[3]H[3]
        \end{bmatrix}
        \begin{bmatrix}
            2   \\
            2   \\
            1   \\
            1
        \end{bmatrix}
        =
        \begin{bmatrix}
            24  \\
            -j2 \\
            0   \\
            j2
        \end{bmatrix}
        \label{eq:L12_S52_1}
    \end{equation}
    and the \( 4 \)-point inverse DFT of \( Y_{C}[k] \) yields:
    \begin{equation}
        \begin{bmatrix}
            y_{C}[0]    \\
            y_{C}[1]    \\
            y_{C}[2]    \\
            y_{C}[3]
        \end{bmatrix}
        =
        \frac{1}{4} \mathbf{D}^{*}_{4}
        \begin{bmatrix}
            Y_{C}[0]    \\
            Y_{C}[1]    \\
            Y_{C}[2]    \\
            Y_{C}[3]
        \end{bmatrix}
        =
        \frac{1}{4}
        \begin{bmatrix}
            1   &   1   &   1   &   1   \\
            1   &   j   &   -1  &   -j  \\
            1   &   -1  &   1   &   -1  \\
            1   &   -j  &   -1  &   j
        \end{bmatrix}
        \begin{bmatrix}
            24  \\
            -j  \\
            0   \\
            j2
        \end{bmatrix}
        =
        \begin{bmatrix}
            6   \\
            7   \\
            6   \\
            5
        \end{bmatrix}
        \label{eq:L12_S53_1}
    \end{equation}
\end{example}

\begin{example}{Circular convolution}{}
    Now, let us extend the two length-\( 4 \) sequences to length-\( 7 \) by appending each with three zero-valued samples, namely:
    \begin{align}
        g_{e}[n]
        &=
            \begin{cases}
                g[n]    &   0 \le n \le 3   \\
                0       &   4 \le n \le 6
            \end{cases} \\
        h_{e}[n]
        &=
            \begin{cases}
                h[n]    &   0 \le n \le 3   \\
                0       &   4 \le n \le 6
            \end{cases}
    \end{align}
    We next determine the \( 7 \)-point circular convolution of \( g_{e}[n] \) and \( h_{e}[n] \):
    \begin{equation}
        y[n]
        =
        \sum_{m=0}^{6} g_{e}[n] h_{e}[\expval{n-m}_{7}],
        \qquad
        0 \le n \le 6
        \label{eq:L12_S55_1}
    \end{equation}
    Therefore:
    \begin{align}
        y[0]
        &=
            g_{e}[0]h_{e}[0] +
            g_{e}[1]h_{e}[6] +
            g_{e}[3]h_{e}[4] +
            g_{e}[4]h_{e}[3] +
            g_{e}[5]h_{e}[2] +
            g_{e}[6]h_{e}[1]    \nonumber   \\
        &=
            g[0]h[0]
            =
            1 \cdot 2
            =
            2
    \end{align}
    Continuing the process, we arrive at:
    \begin{align}
        y[1] &= g_{e}[0]h_{e}[1] + g_{e}[1]h_{e}[0] = 6 \\
        y[2] &= g_{e}[0]h_{e}[2] + g_{e}[1]h_{e}[1] + g_{e}[2]h_{e}[0] = 5  \\
        y[3] &= g_{e}[0]h_{e}[3] + g_{e}[1]h_{e}[2] + g_{e}[2]h_{e}[1] + g_{e}[3]h_{e}[0] = 5   \\
        y[4] &= g_{e}[1]h_{e}[3] + g_{e}[2]h_{e}[2] + g_{e}[3]h_{e}[1] = 4   \\
        y[5] &= g_{e}[2]h_{e}[3] + g_{e}[3]h_{e}[2] = 1 \\
        y[6] &= g_{e}[3]h_{e}[3] = 1
    \end{align}
    So, \( y[n] \) is precisely the sequence \( y_{L}[n] \) obtained by a linear convolution of \( g[n] \) and \( h[n] \).

    \begin{center}
        \includegraphics[width=0.3\textwidth]{\figpath{12}/12_images/S57_1.pdf}
    \end{center}
\end{example}

\medskip
\marginpar{Matrix form of circular convolution and circulant matrix}
In general, the \( N \)-point circular convolution can be written in matrix form as:
\begin{equation}
    \begin{bmatrix}
        y_{C}[0]    \\
        y_{C}[1]    \\
        y_{C}[2]    \\
        \vdots      \\
        y_{C}[N-1]
    \end{bmatrix}
    =
    \begin{bmatrix}
        h[0]    &   h[n-1]  &   \cdots  &   h[1]    \\
        h[1]    &   [0]     &   \cdots  &   h[2]    \\
        h[2]    &   h[1]    &   \cdots  &   h[3]    \\
        \vdots  &   \vdots  &   \ddots  &   \vdots  \\
        1       &   h[n-2]  &   \cdots  &   h[0]
    \end{bmatrix}
    \begin{bmatrix}
        g[0]    \\
        g[1]    \\
        g[2]    \\
        \vdots  \\
        g[N-1]
    \end{bmatrix}
    \label{eq:L12_S58_1}
\end{equation}
Note that the elements of each diagonal of the \( N \times N \) matrix are equal. We call such a matrix a \textbf{circulant matrix}.

\end{document}

\providecommand{\main}{../../main}
\providecommand{\figpath}[1]{\main/../lessons/#1}
\documentclass[../../main/main.tex]{subfiles}



\begin{document}

\chapter{\boldmath The \( z \)-transform}

\newdate{date}{12}{11}{2020}
\marginpar{\textbf{Lecture 14.} \\  \displaydate{date}.}

We have seen that the DTFT provides a frequency-domain representation of discrete-time signals and LTI discrete-time systems. However, because of the convergence condition in many cases, the DTFT of a sequence may not exist. As a result, it is not possible to make use of such frequency-domain characterization in these cases.\marginpar{Problems of DTFT and \( z \)-transform as alternative}

A possible solution and alternative is a generalization of the DTFT, which leads to the \( z \)-transform. The ladder may exist for many sequences for which the DTFT does not exist. Moreover, the use of \( z \)-transform techniques permits simple but powerful algebraic manipulations. Consequently, the \( z \)-transform has become an important tool in the analysis and design of digital filters





\section{\boldmath Definition of \( z \)-transform}
\marginpar{Definition of the \( z \)-transform}
Let us begin the discussion on this topic with the definition of the main tool.
\begin{definition}{\boldmath \( z \)-transform}{}
    For a given sequence \( g[n] \), its \textbf{\boldmath\( z \)-transform} \( G(z) \) is defined as:
    \begin{equation}
        G(z)
        =
        \sum_{n=-\infty}^{\infty} g[n] z^{-n}
        \label{eq:L14_S05_1}
    \end{equation}
    where \( z = \operatorname{Re}[z] + j\operatorname{Im}[z] \) is a complex variable.
\end{definition}

If we let \( z = re^{j\omega} \), then the \( z \)-transform reduces to:
\begin{equation}
    G(re^{j\omega})
    =
    \sum_{n=-\infty}^{\infty} g[n] r^{-n} e^{-j\omega n}
    \label{eq:L14_S06_1}
\end{equation}
\marginpar{Connections to the DTFT}Eq. \ref{eq:L14_S06_1} can be interpreted as the DTFT of the modified sequence \( \qty{g[n]r^{-n}} \). For \( r = 1 \) (i.e., \( \abs{z} = 1 \)), the \( z \)-transform reduces to its DTFT, provided the ladder exists.

Like the DTFT, there are conditions on the convergence of the infinite series like:
\begin{equation}
    \sum_{n=-\infty}^{\infty} g[n] z^{-n}
    \label{eq:L14_S07_1}
\end{equation}
\marginpar{Region of convergence (ROC) of the \( z \)-transform}For a given sequence, the set \( R \) of values of \( z \) for which its \( z \)-transform converges is called the \textbf{region of convergence (ROC)}.

From our earlier discussion on the uniform convergence of the DTFT, it follows that the series:
\begin{equation}
    G(re^{j\omega})
    =
    \sum_{n=-\infty}^{\infty} g[n] r^{-n} e^{-j\omega n}
    \label{eq:L14_S08_1}
\end{equation}
converges if \( \qty{g[n]r^{-n}} \) is absolutely summable, i.e. if:
\begin{equation}
    \sum_{n=-\infty}^{\infty} \abs{g[n] r^{-n}}
    <
    \infty
    \label{eq:L14_S08_2}
\end{equation}
In general, the ROC \( R \) of a \( z \)-transform of a sequence \( g[n] \) is an annular region of the \( z \)-plane, namely:
\begin{equation}
    R_{g^{-}}
    <
    \abs{z}
    <
    R_{g^{+}}
    \label{eq:L14_S09_1}
\end{equation}
where \( 0 \le R_{g^{-}} < R_{g^{+}} < \infty \).


\medskip
After having introduced the argument, we study some examples of \( z \)-transform calculation in order to understand how it works in practice.\marginpar{Examples of \( z \)-transform calculation}
\begin{example}{\boldmath\( z \)-transform calculation}{}
    We determine the \( z \)-transform \( X(z) \) of the causal sequence \( x[n] = \alpha^{n} \mu[n] \) and its ROC. Now:
    \begin{equation}
        X(z)
        =
        \alpha^{n} \mu[n] z^{-n}
        =
        \sum_{n=0}^{\infty} \alpha^{n} z^{-n}
        \label{eq:L14_S10_1}
    \end{equation}
    The above power series converges to:
    \begin{equation}
        X(z)
        =
        \frac{1}{1 - \alpha z^{-1}}
        \qquad
        \abs{\alpha z^{-1}} < 1
        \label{eq:L14_S10_2}
    \end{equation}
    Therefore, the ROC is the annular region \( \abs{z} > \abs{\alpha} \).
\end{example}

\begin{example}{\boldmath\( z \)-transform calculation}{}
    The \( z \)-transform \( \mu(z) \) of the unit step sequence \( \mu[n] \) can be obtained from:
    \begin{equation}
        X(z)
        =
        \frac{1}{1 - \alpha z^{-1}}
        \qquad
        \abs{\alpha z^{-1}} < 1
        \label{eq:L14_S11_1}
    \end{equation}
    By setting \( \alpha = 1 \):
    \begin{equation}
        \mu(z)
        =
        \frac{1}{1 - z^{-1}}
        \qquad
        \abs{z^{-1}} < 1
        \label{eq:L14_S11_2}
    \end{equation}
    Therefore, the ROC is the annular region \( 1 < \abs{z} < \infty \). Note that the unit step sequence \( \mu[n] \) is not obsolutely summable, and hence its DTFT does not converge uniformly.
\end{example}

\begin{example}{\boldmath\( z \)-transform calculation}{}
    Consider the anti-causal sequence:
    \begin{equation}
        y[n]
        =
        - \alpha^{n} \mu[-n-1]
        \label{eq:L14_S12_1}
    \end{equation}
    Its \( z \)-transform is given by:
    \begin{align}
        Y(z)
        &=
            - \sum_{n=-\infty}^{-1} \alpha^{n} z^{-n}
            =
            - \sum_{m=1}^{\infty} \alpha^{-m} z^{m} \nonumber   \\
        &=
            - \alpha^{-1} z \sum_{m=0}^{\infty} \alpha^{-m} z^{m}
            =
            - \frac{- \alpha^{-1} z}{1 - \alpha z^{-1}} \nonumber   \\
        &=
            \frac{1}{1 - \alpha z^{-1}}
    \end{align}
    for \( \abs{\alpha^{-1}z} < 1 \). Therefore, the ROC is the annular region \( \abs{z} < \abs{\alpha} \).
\end{example}

Note that the \( z \)-transforms of the two sequences \( \alpha^{n} \mu[n] \) and \( - \alpha^{n} \mu[-n-1] \) are identical even though the two parent sequences are different. The only way a unique sequence can be associated with a \( z \)-transform is by specifying its ROC.

\marginpar{Connection between uniform convergence of the DTFT and the ROC}
Another important point is that the DTFT \( G(e^{j\omega}) \) of a sequence \( g[n] \) converges uniformly if and only if the ROC of the \( z \)-transform \( G(z) \) of \( g[n] \) includes the unit circle. However, the existence of the DTFT does not always imply the existence of the \( z \)-transform.

\begin{example}{\boldmath\( z \)-transform calculation}{}
    The finite energy sequence:
    \begin{equation}
        h_{LP}[n]
        =
        \frac{\sin\qty(\omega_{c}n)}{\pi n}
        \qquad
        - \infty < n < \infty
        \label{eq:L14_S15_1}
    \end{equation}
    has a DTFT given by:
    \begin{equation}
        H_{LP}(e^{j\omega})
        =
        \begin{cases}
            1   &   0 \le \abs{\omega} \le \omega_{c}   \\
            0   &   \omega_{c} < \abs{\omega} \le \pi
        \end{cases}
        \label{eq:L14_S15_2}
    \end{equation}
    which converges in the mean-square sense. However, \( h_{LP}[n] \) does not have a \( z \)-transform as it is not absolutely summable for any value of \( r \).
\end{example}


\medskip
Some commonly used \( z \)-transform pairs are listed in Figure \ref{fig:L14_S16_1}.\marginpar{Commonly used \( z \)-transform pairs}

\begin{figure}[!h]
    \centering
    \includegraphics[width=0.75\textwidth]{\figpath{14}/14_images/S16_1.pdf}
    \caption{\label{fig:L14_S16_1} Common \( z \)-transform pairs.}
\end{figure}





\section{Rational \( z \)-transforms}
\marginpar{Rational \( z \)-transforms}
In the case of the LTI discrete-time systems we are concerned with in this course, all the pertinent \( z \)-transforms are \textbf{rational functions} of \( z^{-1} \), that is, they are fractions of two polinomials in \( z^{-1} \):
\begin{equation}
    G(z)
    =
    \frac{P(z)}{D(z)}
    =
    \frac{p_{0} + p_{1}z^{-1} + \dots + p_{M-1}z^{-(M-1)} + p_{M}z^{-M}}{d_{0} + d_{1}z^{-1} + \dots + d_{N-1}z^{-(N-1)} + d_{N}z^{-N}}
    \label{eq:L14_S17_1}
\end{equation}
The degree of the numerator polynomial \( P(z) \) is \( M \) and the degree of the denominator polynomial \( D(z) \) is \( N \). An alternate representation of a rational \( z \)-transform is as a ratio of two polynomials in \( z \):
\begin{equation}
    G(z)
    =
    z^{(N-M)} \frac{p_{0}z^{M} + \dots + p_{M-1}z + p_{M}}{d_{0}z^{N} + \dots + d_{N-1}z + d_{N}}
    \label{eq:L14_S18_1}
\end{equation}
Again, a rational \( z \)-transform can be alternately written in \textbf{factored form} as:\marginpar{Rational \( z \)-transform in factored form}
\begin{equation}
    G(z)
    =
    \frac{\displaystyle p_{0} \prod_{\ell=1}^{M} (1 - \xi_{\ell}z^{-1})}{\displaystyle d_{0} \prod_{\ell=1}^{N} (1 - \lambda_{\ell}z^{-1})}
    =
    z^{(N-M)} \frac{\displaystyle p_{0} \prod_{\ell=1}^{M} (z - \xi_{\ell})}{\displaystyle d_{0} \prod_{\ell=1}^{N} (z - \lambda_{\ell})}
    \label{eq:L14_S19_1}
\end{equation}
In particular, we have the following quatities of interest:
\marginpar{Zeros and poles of rational \( z \)-transform}
\begin{itemize}
    \item \( z = \xi_{\ell} \), roots of the numerator polynomial. These values of \( z \) are known as the \textbf{zeros} of \( G(z) \);
    \item \( z = \lambda_{\ell} \), {roots} of the denominator polynomial. These values of \( z \) are known as the \textbf{poles} of \( G(z) \).
\end{itemize}

\begin{example}{\boldmath ROC of a rational \( z \)-transform}{}
    The \( z \)-transform \( H(z) \) of the sequence \( h[n] = \qty(-0.6)^{n} \mu[n] \) is given by:
    \begin{equation}
        H(z)
        =
        \frac{1}{1 + 0.6z^{-1}}
        \qquad
        \abs{z} > 0.6
        \label{eq:L14_S23_1}
    \end{equation}
    Here the ROC is just outside the circle going through the point \( z = - 0.6 \).
\end{example}

\begin{example}{Zeros and poles}{}
    The \( z \)-transform:
    \begin{equation}
        \mu(z)
        =
        \frac{1}{1 - z^{-1}}
        \qquad
        \abs{z} > 1
        \label{eq:L14_S22_1}
    \end{equation}
    has a zero at \( z = 0 \) and a pole at \( z = 1 \).
\end{example}

A physical interpretation of the concepts of poles and zeros can be given by plotting the \textbf{log-magnitude} \( 20\log_{10}\abs{G(z)} \) as showed in Figure \ref{fig:L14_S25_1} for:
\marginpar{Log-magnitude plot of \( z \)-transform for physical interpretation}
\begin{equation}
    G(z)
    =
    \frac{1 - 2.4z^{-1} + 2.88z^{-2}}{1 - 0.8z^{-1} + 0.64z^{-2}}
    \label{eq:L14_S24_1}
\end{equation}
Observe that the \textbf{magnitude plot} exhibits very large peaks around the points \( z = 0.4 \pm j0.6928 \), which are the poles of \( G(z) \). It also exhibits very narrow and deep wells around the location of the zeros at \( z = 1.2 \pm j1.2 \).

\begin{figure}[!h]
    \centering
    \includegraphics[width=0.5\textwidth]{\figpath{14}/14_images/S25_1.pdf}
    \caption{\label{fig:L14_S25_1} Log-magnitude plot for \( G(z) \) in Eq. \ref{eq:L14_S24_1}.}
\end{figure}


\medskip
Now, we remark that the ROC of a \( z \)-transform is an important concept.
\marginpar{Importance of the ROC of \( z \)-transform}
Without its knowledge, there is no unique relationship between a sequence and its \( z \)-transform. Hence, \textbf{\boldmath the \( z \)-transform must always be specified with its ROC}. Moreover, there is a relationship between the ROC of the \( z \)-transform of the impulse response of a causal LTI discrete-time system and its BIBO stability.

Another important distiction is that a sequence can be one of the following types:
\marginpar{ROC dependency on the type of sequence}
\begin{itemize}
    \item \textbf{finite-length};
    \item \textbf{right-sided};
    \item \textbf{left-sided};
    \item \textbf{two-sided}.
\end{itemize}
In general, the ROC depends on the type of the sequence of interest and we show it in the following Subsections.



\subsection{\boldmath Finite-length sequence \( z \)-transform}
\marginpar{\( z \)-transform of finite-length sequence}
We consider a \textbf{finite-length sequence} \( g[n] \) defined for \( -M \le n \le N \), where \( M \) and \( N \) are non-negative integers and \( \abs{g[n]} < \infty \).
Its \( z \)-transform is given by:
\begin{equation}
    G(z)
    =
    \sum_{n=-M}^{N} g[n] z^{-n}
    =
    \frac{\displaystyle \sum_{n=0}^{N+M} g[n-M]z^{N+M-n}}{z^{N}}
    \label{eq:L14_S31_1}
\end{equation}
Note that \( G(z) \) has \( M \) zeros and \( N \) poles. As can be seen from the expression for \( G(z) \), the \( z \)-transform of a finite-length bounded sequence converges everywhere in the z-plane except possibly at \( z = 0 \) and/or at \( z = \infty \).



\subsection{\boldmath Right-sided sequence \( z \)-transform}
\marginpar{\( z \)-transform of right-sided sequence}
A \textbf{right-sided sequence} with nonzero sample values for \( n \ge 0 \) is sometimes called a \textbf{causal sequence}. So, we consider a causal sequence \( u_{1}[n] \). Its \( z \)-transform is given by:
\begin{equation}
    U_{1}(z)
    =
    \sum_{n=0}^{\infty} u_{1}[n] z^{-n}
    \label{eq:L14_S32_1}
\end{equation}
It can be showed that \( U_{1}(z) \) converges exterior to a circle with \( \abs{z} = R_{1} \), including the point \( z = \infty \).

On the other hand, a right-sided sequence \( u_{2}[n] \) with nonzero sample values only for \( n \ge -M \) with \( M \) non-negative has a \( z \)-transform \( U_{2}(z) \) with \( M \) poles at \( z = \infty \).
The ROC of \( U_{2}(z) \) is exterior to a circle \( \abs{z} = R_{2} \), exluding the point \( z = \infty \).



\subsection{\boldmath Left-sided sequence \( z \)-transform}
\marginpar{\( z \)-transform of left-sided sequence}
A \textbf{left-sided sequence} with nonzero sample values for \( n \le 0 \) is sometimes called \textbf{anticausal sequence}. So, we consider an anticausal sequence \( v_{1}[n] \). Its \( z \)-transform is given by:
\begin{equation}
    V_{1}(z)
    =
    \sum_{n=-\infty}^{0} v_{1}[n]z^{-n}
    \label{eq:L14_S34_1}
\end{equation}
It can be showed that \( V_{1}(z) \) converges interior to a circle \( \abs{z} = R_{3} \), including the point \( z = 0 \).

On the other hand, a left-sided sequence with nonzero sample values only for \( n \le N \) with \( N \) non-negative has a \( z \)-transform \( V_{2}(z) \) with \( N \) poles at \( z = 0 \).
The ROC of \( V_{2}(z) \) is interior to a circle \( \abs{z} = R_{4} \), excluding the point \( z = 0 \).



\subsection{\boldmath Two-sided sequence \( z \)-transform}
\marginpar{\( z \)-transform of two-sided sequence}
The \( z \)-transform of a \textbf{two-sided sequence} \( w[n] \) can be expressed as:
\begin{equation}
    W(z)
    =
    \sum_{n=-\infty}^{\infty} w[n]z^{-n}
    =
    \sum_{n=0}^{\infty} w[n]z^{-n} + \sum_{n=-\infty}^{-1} w[n]z^{-n}
    \label{eq:L14_S36_1}
\end{equation}
The first term on the RHS can be interpreted as the \( z \)-transform of a right-sided sequence and it thus converges exterior to the circle \( \abs{z} = R_{5} \). The second term of the RHS can be interpreted as the \( z \)-transform of a left-sided sequence and it thus converges interior to the circle \( \abs{z} = R_{6} \).
If \( R_{5} < R_{6} \), there is an overlapping ROC given  by \( R_{5} < \abs{z} < R_{6} \). If \( R_{5} > R_{6} \), there is no overlap and the \( z \)-transform does not exist.

In particular, let us consider as example the two-sided sequence:
\begin{equation}
    u[n]
    =
    \alpha^{n}
    \label{eq:L14_S38_1}
\end{equation}
where \( \alpha \) can be either real or complex. Its \( z \)-transform is given by:
\begin{equation}
    U(z)
    =
    \sum_{n=-\infty}^{\infty} \alpha^{n}z^{-n}
    =
    \sum_{n=0}^{\infty} \alpha^{n}z^{-n} + \sum_{n=-\infty}^{-1} \alpha^{n}z^{-n}
    \label{eq:L14_S38_2}
\end{equation}
The first term on the RHS converges for \( \abs{z} > \abs{\alpha} \), whereas the second term converges for \( \abs{z} < \abs{\alpha} \). There is no overlap between these two regions, hence the \( z \)-transform of \( u[n] = \alpha^{n} \) does not exist.


\medskip
The ROC of a rational \( z \)-transform cannot contain any pole (since it is infinite at a pole) and is bounded by the poles.
To show the latter statement, we assume that the \( z \)-transform \( X(z) \) has simple poles at \( z = \alpha \) and \( z = \beta \). We also assume that the corresponding sequence \( x[n] \) is a right-sided sequence.
Then, \( x[n] \) has the form:
\begin{equation}
    x[n]
    =
    \qty(r_{1}\alpha^{n} + r_{2}\beta^{n}) \mu[n-N_{0}]
    \qquad
    \abs{\alpha} < \abs{\beta}
    \label{eq:L14_S41_1}
\end{equation}
where \( N_{0} \) is a positive or negative integer. Now, the \( z \)-transform of the right-sided sequence \( \gamma^{n} \mu[n-N_{0}] \) exists if:
\begin{equation}
    \sum_{n=N_{0}}^{\infty} \abs{\gamma^{n}z^{-n}}
    <
    \infty
    \label{eq:L14_S41_2}
\end{equation}
for some \( z \). The condition in Eq. \ref{eq:L14_S41_2} holds for \( \abs{z} > \abs{\gamma} \), but not for \( \abs{z} \le \abs{\gamma} \). Therefore, the \( z \)-transform of Eq. \ref{eq:L14_S41_1} has a ROC defined by \( \abs{\beta} < \abs{z} \le \infty \).

Likewise, the \( z \)-transform of a left-sided sequence:
\begin{equation}
    x[n]
    =
    \qty(r_{1}\alpha^{n} + r_{2}\beta^{n}) \mu[-n-N_{0}]
    \qquad
    \abs{\alpha} < \abs{\beta}
    \label{eq:L14_S43_1}
\end{equation}
has a ROC defined by \( 0 \le \abs{z} < \abs{\alpha} \).





\section{\boldmath Inverse \( z \)-transform}
\marginpar{Inverse \( z \)-transform using DTFT analogy}
Firstly, we recall that, for \( z = re^{j\omega} \), the \( z \)-transform \( G(z) \) given by:
\begin{equation}
    G(z)
    =
    \sum_{n=-\infty}^{\infty} g[n]z^{-n}
    =
    \sum_{n=-\infty}^{\infty} g[n]r^{-n}e^{-j\omega n}
    \label{eq:L14_S47_1}
\end{equation}
is the DTFT of the modified sequence \( g[n]r^{-n} \). Accordingly, the inverse DTFT is thus given by:
\begin{equation}
    g[n]r^{-n}
    =
    \frac{1}{2\pi} \int_{-\pi}^{\pi} G(re^{j\omega})e^{j\omega n} \d{\omega}
    \label{eq:L14_S47_2}
\end{equation}
\marginpar{Inverse \( z \)-transform through contour integral}By making a change of variable \( z = re^{j\omega} \), the previous equation can be converted into a \textbf{contour integral} given by:
\begin{equation}
    g[n]
    =
    \frac{1}{2\pi j} \oint_{C'} G(z)z^{n-1} \d{z}
    \label{eq:L14_S48_1}
\end{equation}
where \( C' \) is a counterclockwise contour of integration defined by \( \abs{z} = r \).
But the integral remains unchanged when it is replaced with any contour \( C \) encircling the point \( z = 0 \) in the ROC of \( G(z) \). The contour integral can be evaluated using the \textbf{Cauchy's residue theorem} resulting in:\marginpar{Cauchy's residue theorem}
\begin{equation}
    g[n]
    =
    \sum \underset{C}{\operatorname{Res}}\qty[G(z)z^{n-1}]
    \label{eq:L14_S49_1}
\end{equation}
Eq. \ref{eq:L14_S49_1} needs to be evaluated at all the values of \( n \), but this is not pursued here.


\medskip
Now, a rational \( z \)-transform \( G(z) \) with a causal inverse transform \( g[n] \) has an ROC that is exterior to a circle.
\marginpar{\( z \)-transform in partial fraction expansion form}
Here, it is more convenient to express \( G(z) \) in a \textbf{partial-fraction expansion form} and then determine \( g[n] \) by summing the inverse transforms of the individual simpler terms in the expansion. Therefore, a rational \( G(z) \) can be expressed as:
\begin{equation}
    G(z)
    =
    \frac{P(z)}{D(z)}
    =
    \frac{\displaystyle \sum_{i=0}^{M} p_{i}z^{-i}}{\displaystyle \sum_{i=0}^{N} d_{i}z^{-i}}
    \label{eq:L14_S51_1}
\end{equation}
If \( M \ge N \), then \( G(z) \) can be re-expressed as:
\begin{equation}
    G(z)
    =
    \sum_{\ell=0}^{M-N} \eta_{\ell}z^{-\ell} + \frac{P_{1}(z)}{D(z)}
    \label{eq:L14_S51_2}
\end{equation}
where the degree of \( P_{1}(z) \) is less than \( N \). The rational function \( \frac{P_{1}(z)}{D(z)} \) is called a \textbf{proper fraction}.
\marginpar{Proper fractions and long division technique}
To develop the proper fraction part \( \frac{P_{1}(z)}{D(z)} \) from \( G(z) \), a \textbf{long division} of \( P(z) \) by \( D(z) \) should be carried out in a reverse order until the remainder polynomial \( P_{1}(z) \) is of lower degree that that of the denominator \( D(z) \).

\begin{example}{Inverse transform by partial-fraction expansion}{}
    Consider:
    \begin{equation}
        G(z)
        =
        \frac{2 + 0.8z^{-1} + 0.5z^{-2} + 0.3z^{-3}}{1 + 0.8z^{-1} + 0.2z^{-2}}
        \label{eq:L14_S53_1}
    \end{equation}
    By long division in reverse order we arrive at:
    \begin{equation}
        G(z)
        =
        -3.5 + 1.5z^{-1} + \underbrace{\frac{5.5 + 2.1z^{-1}}{1 + 0.8z^{-1} + 0.2z^{-2}}}_{\text{Proper fraction}}
        \label{eq:L14_S53_2}
    \end{equation}
\end{example}

In most practical cases, the rational \( z \)-transform of interest \( G(z) \) is a proper fraction with simple poles. Let the poles of \( G(z) \) be at \( z = \lambda_{k} \), with \( 1 \le k \le N \). A partial-fraction expansion of \( G(z) \) is then of the form:
\begin{equation}
    G(z)
    =
    \sum_{\ell=1}^{N} \qty(\frac{\rho_{\ell}}{1 - \lambda_{\ell}z^{-1}})
    \label{eq:L14_S54_1}
\end{equation}
The constants \( \rho_{\ell} \) in the partial-fraction expansion are called the \textbf{residues} and are given by:
\marginpar{Residues in partial-fraction expansion}
\begin{equation}
    \rho_{\ell}
    =
    \qty[(1 - \lambda_{\ell}z^{-1}) G(z)]_{z=\lambda_{\ell}}
    \label{eq:L14_S55_1}
\end{equation}
Each term of the sum in partial-fraction expansion has a ROC given by \( \abs{z} > \abs{\lambda_{\ell}} \) and thus has an inverse transform of the form \( \rho_{\ell}(\lambda_{\ell})^{n} \mu[n] \).
Therefore, the inverse transform \( g[n] \) of \( G(z) \) is given by:
\begin{equation}
    g[n]
    =
    \sum_{\ell=1}^{N} \rho_{\ell} (\lambda_{\ell})^{n} \mu[n]
    \label{eq:L14_S56_1}
\end{equation}
Note that the approach in Eq. \ref{eq:L14_S56_1} with a slight modification can also be used to determine the inverse of a rational \( z \)-transform of a non-causal sequence.

\begin{example}{Inverse transfrom of a causal sequence}{}
    Let the \( z \)-transform \( H(z) \) of a causal sequence \( h[n] \) be given by:
    \begin{equation}
        H(z)
        =
        \frac{z(z+2)}{(z-0.2)(z+0.6)}
        =
        \frac{1 + 2z^{-1}}{(1-0.2z^{-1})(1 + 0.6z^{-1})}
        \label{eq:L14_S57_1}
    \end{equation}
    A partial-fraction expansion of \( H(z) \) is then of the form:
    \begin{equation}
        H(z)
        =
        \frac{\rho_{1}}{1 - 0.2z^{-1}} + \frac{\rho_{2}}{1 - 0.6z^{-1}}
        \label{eq:L14_S57_2}
    \end{equation}
    Now:
    \begin{align}
        \rho_{1} &= \qty[(1-0.2z^{-1})H(z)]_{z=0.2} = \qty[\frac{1+2z^{-1}}{1+0.6z^{-1}}]_{z=0.2} = 2.75    \\
        \rho_{2} &= \qty[(1+0.6z^{-1})H(z)]_{z=-0.6} = \qty[\frac{1+2z^{-1}}{1-0.2z^{-1}}]_{z=-0.6} = -1.75
    \end{align}
    Hence:
    \begin{equation}
        H(z)
        =
        \frac{2.75}{1 - 0.2z^{-1}} - \frac{1.75}{1 + 0.6z^{-1}}
        \label{eq:L14_S59_1}
    \end{equation}
    The inverse transform of Eq. \ref{eq:L14_S59_1} is therefore given by:
    \begin{equation}
        h[n]
        =
        2.75(0.2)^{n} \mu[n] - 1.75(-0.6)^{n} \mu[n]
        \label{eq:L14_S59_2}
    \end{equation}
\end{example}

In case \( G(z) \) has multiple poles, \marginpar{Partial-fraction expansion with poles of higher multiplicity}the partial-fraction expansion is of slightly different form.
Let the pole at \( z = v \) be of multiplicity \( L \) and the remaining \( N - L \) poles be simple and at \( z = \lambda_{\ell} \), for \( 1 \le \ell \le N-L \).
Then, the partial-fraction expansion of \( G(z) \) is of the form:
\begin{equation}
    G(z)
    =
    \sum_{\ell=0}^{M-N} \eta_{\ell}z^{-\ell} + \sum_{\ell=1}^{N-L} \frac{\rho_{\ell}}{1 - \lambda_{\ell}z^{-1}} + \sum_{i=1}^{L} \frac{\gamma_{i}}{(1 - vz^{-1})^{i}}
    \label{eq:L14_S61_1}
\end{equation}
where the constants \( \gamma_{i} \) are computed using:
\begin{equation}
    \gamma_{i}
    =
    \frac{1}{(L-i)! (-v)^{L-i}} \dv[L-i]{}{(z^{-1})} \qty[(1-vz^{-1})G(z)]_{z=v}
    \qquad
    1 \le i \le L
    \label{eq:L14_S61_2}
\end{equation}
The residues \( \rho_{\ell} \) are calculated as before.





\section{\boldmath\( z \)-transform properties}
A list of properties of the \( z \)-transform is showed in Figure \ref{fig:L14_S64_1}.

\marginpar{Properties of \( z \)-transform}
\begin{figure}[!h]
    \centering
    \includegraphics[width=1\textwidth]{\figpath{14}/14_images/S64_1.pdf}
    \caption{\label{fig:L14_S64_1} Properties of \( z \)-transform.}
\end{figure}

\marginpar{Examples for \( z \)-transform properties}
Now, we present some examples with cases where the properties can be usefully applied.

\begin{example}{\boldmath\( z \)-transform properties}{}
    Consider the two-sided sequence:
    \begin{equation}
        v[n]
        =
        \alpha^{n} \mu[n] - \beta^{n} \mu[-n-1]
        \label{eq:L14_S65_1}
    \end{equation}
    Let \( x[n] = \alpha^{n} \mu[n] \) and \( y = - \beta^{n} \mu[-n-1] \) with \( X(z) \) and \( Y(z) \) denoting, respectively, their \( z \)-transforms. Now:
    \begin{align}
        X(z) &= \frac{1}{1 - \alpha z^{-1}}   &   \abs{z} > \abs{\alpha}  \\
        Y(z) &= \frac{1}{1 - \beta z^{-1}}  &   \abs{z} < \abs{\beta}
    \end{align}
    Using the linearity property we arrive at:
    \begin{equation}
        V(z)
        =
        X(z) + Y(z)
        =
        \frac{1}{1 - \alpha z^{-1}} + \frac{1}{1 - \beta z^{-1}}
        \label{eq:L14_S66_1}
    \end{equation}
    The ROC of \( V(z) \) is given by the overlap regions of \( \abs{z} > \abs{\alpha} \) and \( \abs{z} < \abs{\beta} \). We have that:
    \begin{itemize}
        \item if \( \abs{\alpha} < \abs{\beta} \), then there is an overlap and the ROC is an annular region \( \abs{\alpha} < \abs{z} < \abs{\beta} \);
        \item if \( \abs{\alpha} > \abs{\beta} \), then there is no overlap and \( V(z) \) does not exist.
    \end{itemize}
\end{example}

\begin{example}{\boldmath\( z \)-transform properties}{}
    We determine the \( z \)-transform and its ROC of the causal sequence:
    \begin{equation}
        x[n]
        =
        r^{n}(\cos\qty(\omega_{0}n)) \mu[n]
        \label{eq:L14_S67_1}
    \end{equation}
    We can express \( x[n] = v[n] + v^{*}[n] \), where:
    \begin{equation}
        v[n]
        =
        \frac{1}{2} r^{n} e^{j\omega_{0}n} \mu[n] = \frac{1}{2} \alpha^{n} \mu[n]
        \label{eq:L14_S67_2}
    \end{equation}
    The \( z \)-transform of \( v[n] \) is given by:
    \begin{equation}
        V(z)
        =
        \frac{1}{2} \frac{1}{1 - \alpha z^{-1}}
        =
        \frac{1}{2} \frac{1}{1 - re^{j\omega_0 z^{-1}}}
        \qquad
        \abs{z} > \abs{\alpha} = r
        \label{eq:L14_S67_3}
    \end{equation}
    Using the conjugation property, we obtain the \( z \)-transform of \( v^{*}[n] \) as:
    \begin{equation}
        V^{*}(z^{*})
        =
        \frac{1}{2} \frac{1}{1 - \alpha^{*} z^{-1}}
        =
        \frac{1}{2} \frac{1}{1 - re^{-j\omega_{0}z^{-1}}}
        \qquad
        \abs{z} > \abs{\alpha}
        \label{eq:L14_S68_1}
    \end{equation}
    Finally, using the linearity property we get:
    \begin{equation}
        X(z)
        =
        V(z) + V^{*}(z^{*})
        =
        \frac{1}{2} \qty(\frac{1}{1 - re^{j\omega_0 z^{-1}}} + \frac{1}{1 - re^{-j\omega_0 z^{-1}}})
        \label{eq:L14_S68_2}
    \end{equation}
    or:
    \begin{equation}
        X(z)
        =
        \frac{1 - (r\cos\omega_{0})z^{-1}}{1 - (2r\cos\omega_{0})z^{-1} + r^{2}z^{-2}}
        \qquad
        \abs{z} > r
        \label{eq:L14_S68_3}
    \end{equation}
\end{example}

\begin{example}{\( z \)-transform properties}{}
    We determine the \( z \)-transform \( Y(z) \) and the ROC of the sequence:
    \begin{equation}
        y[n]
        =
        (n+1) \alpha^{n} \mu[n]
        \label{eq:L14_S69_1}
    \end{equation}
    We can write \( y[n] = nx[n] + x[n] \) where:
    \begin{equation}
        x[n]
        =
        \alpha^{n} \mu[n]
        \label{eq:L14_S69_2}
    \end{equation}
    Now, the \( z \)-transform \( X(z) \) of \( x[n] = \alpha^{n} \mu[n] \) is given by:
    \begin{equation}
        X(z)
        =
        \frac{1}{1 - \alpha z^{-1}}
        \qquad
        \abs{z} > \abs{\alpha}
        \label{eq:L14_S70_1}
    \end{equation}
    Using the differentiation property, we arrive at the \( z \)-transform of \( nx[n] \) as:
    \begin{equation}
        -z \dv{X(z)}{z}
        =
        \frac{\alpha z^{-1}}{1 - \alpha z^{-1}}
        \qquad
        \abs{z} > \abs{\alpha}
        \label{eq:L14_S70_2}
    \end{equation}
    Using the linearity property we finally obtain:
    \begin{equation}
        Y(z)
        =
        \frac{1}{1 - \alpha z^{-1}} + \frac{\alpha z^{-1}}{(1 - \alpha z^{-1})^2}
        =
        \frac{1}{(1 - \alpha z^{-1})^2}
        \qquad
        \abs{z} > \abs{\alpha}
        \label{eq:L14_S71_1}
    \end{equation}
\end{example}

\end{document}
